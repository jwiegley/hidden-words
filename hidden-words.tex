% -*- bidi-paragraph-direction: left-to-right -*-

\documentclass[11pt]{article}

\usepackage[margin=1in,top=1.25in,bottom=1.25in]{geometry}
\usepackage{soul}
\usepackage{tabularx}
\usepackage{fontspec}
\usepackage{xunicode}
\usepackage{setspace}
\usepackage{arabxetex}

\setuldepth{sh}

\setmainfont[Ligatures=TeX]{GaramondPremrPro}[
  Path           = /Users/johnw/Library/Fonts/ ,
  Extension      = .otf ,
  BoldFont       = *-Smbd ,
  ItalicFont     = *-It ,
  BoldItalicFont = *-SmbdIt
]

\newfontfamily\arabicfont{Scheherazade}[
  Path        = /Users/johnw/Library/Fonts/ ,
  Extension   = .ttf ,
  UprightFont = *RegOT ,
  Script      = Arabic
]

\newfontfamily\headwordfont[Ligatures=TeX]{Georgia}[
  Path           = /Users/johnw/Library/Fonts/ ,
  Extension      = .ttf ,
  UprightFont    = * ,
  BoldFont       = * Bold ,
  ItalicFont     = * Italic ,
  BoldItalicFont = * Bold Italic ,
  Script         = Arabic
]

\newenvironment{orig}
  {\begin{farsi}[voc]}
  {\end{farsi}}

\newenvironment{trans}
  {\Large\begin{spacing}{1.2}\raggedright}
  {\end{spacing}}

\newenvironment{word}
  {\begin{tabular}[t]{p{2.75in}@{\hspace{3em}}p{2.75in}}}
  {\end{tabular}}

\newcommand{\ayat}[2]{\begin{orig}#1\end{orig} & \begin{trans}#2\end{trans}}
\newcommand{\heading}[2]{\textsc{\textbf{#1}} % \ (\##2)
}
\newcommand{\define}[3]{\textfarsi[voc]{\Huge
    \textbf{#1}}\hspace{3mm}{\headwordfont \large
    \textit{#2}}\hspace{3mm}{\Large #3} \\[3ex]}
\newcommand{\fulldefine}[3]{\textfarsi[voc]{\Huge
    \textbf{#1}}\hspace{3mm}{\headwordfont \large
    \textit{#2}}\vspace{-1ex}\begin{quote}\Large #3\end{quote}\vspace{1ex}}

\title{
\Huge
\vspace*{2in}
The Hidden Words \\
\vspace{.25in}
\fontsize{48}{36}
\begin{arab}
کلمات مکنونه
\end{arab}
\vspace{1in}}
\author{\LARGE by Bahá’u’lláh}
\date{}

\begin{document}

\maketitle
\thispagestyle{empty}

\newpage

\fontsize{24}{32}

\vspace*{3in}

\begin{word}
\ayat{
به نام گویندۀ توانا
}{
In the Name of the Lord of Utterance, The Mighty.
}
\end{word}

\pagebreak

\begin{word}
\ayat{ای صاحبانِ هوش و گوش}{\heading{
  O ye People that have Minds to Know and Ears to Hear!}{}} \\ \ayat{
اوّل سروشِ دوست اینست
}{The first call of the Beloved is this:} \vspace{-1ex}\\ \ayat{
ای بلبلِ معنوی

جز در گُلبُنِ معانی جای مَگُزين
}{O mystic nightingale! Abide not but in the rose-garden of the spirit.} \\ \ayat{
و ای هدهدِ سلیمانِ عشق

جز در سبایِ جانان وطن مگیر
}{O messenger of the Solomon of love! Seek thou no shelter except in the Sheba
  of the well-beloved,} \vspace{-1ex}\\ \ayat{
و ای عَنقایِ بقا

جز در قافِ وفا محلّ مپذیر
}{and O immortal phoenix! dwell not save on the mount of faithfulness.} \\ \ayat{
اینست مکانِ تو

اگر بِلامَکان بِپَرِ جان برپری

و آهنگِ مقامِ خود رایگان نمائی
}{Therein is thy habitation, if on the wings of thy soul thou soarest to the
  realm of the infinite and seekest to attain thy goal.}
\end{word}

\newpage

\begin{minipage}[t]{0.48\textwidth}
\define{صاحبان}{sá\d{h}ibán}{owners}
\define{هوش}{hú\ul{sh}}{mind}
\define{گوش}{gú\ul{sh}}{ear}
\define{سروش}{surú\ul{sh}}{song}
\define{بلبل}{bulbul}{nightingale}
\define{معنوی}{ma`naví}{}
\define{گُلبُن}{gulbun}{}
\define{معانی}{ma`ání}{meaning}
\define{مَگُزين}{maguzín}{}
\fulldefine{هدهد}{hudhud}{The hoopoe bird, often a guide in Sufi literature.}
\define{سلیمان}{Sulaymán}{King Solomon}
\define{عشق}{`i\ul{sh}q}{love}
\define{سبا}{Sabá}{Sheba}
\end{minipage}
\begin{minipage}[t]{0.48\textwidth}
\define{جانان}{jánán}{The Beloved}
\define{وطن}{va\d{t}an}{homeland}
\define{عَنقا}{`anqá}{phoenix}
\define{بقا}{baqá}{eternity}
\fulldefine{قاف}{Qáf}{The fabled mount Qáf, thought to symbolize the Holy Qur'án.}
\define{وفا}{vafá}{faithfulness}
\define{محلّ}{ma\d{h}all}{place}
\define{مپذیر}{mapa\ul{dh}ír}{}
\define{مکان}{makán}{place}
\define{بِلامَکان}{bi-lá-makán}{lit.~``to no place''}
\define{آهنگ}{áhang}{}
\fulldefine{مقام}{maqám}{Station, often used in Sufism to denote a degree of
  spiritual understanding.}
\define{رایگان نمودن}{ráygán namúdan}{}
\end{minipage}

\newpage

\begin{word}
\ayat{ای پسرِ روح}{\heading{O Son of Spirit!}{}} \\ \ayat{
هر طیری را نظر بر آشیان است

و هر بلبلی را مقصودْ جمالِ گُل
}{The bird seeketh its nest; the nightingale the charm of the rose;} \\ \ayat{
مگر طُیورِ اَفَئدهِ عباد

که بِتُرابِ فانی قانع شده

از آشیانِ باقی دور مانده اند
}{whilst those birds, the hearts of men, content with transient dust, have
  strayed far from their eternal nest,} \\\vspace{-2ex} \ayat{
و بگِلهایِ بُعد توجّه نموده

از گُلهایِ قُرب محروم گشته اند}
{and with eyes turned towards the slough of heedlessness are bereft of the
  glory of the divine presence.} \vspace{-1ex}\\ \ayat{
زِهی حیرت و حَسرت

و اَفسوس و دَریغ
}{Alas! How strange and pitiful;} \\ \ayat{
که بِاِبريقی

از اَمواجِ بحرِ رفیقِ اَعلیٰ گُذَشته اند

و از اُفُقِ ابهیٰ دور مانده اند
}{for a mere cupful, they have turned away from the billowing seas of the Most
  High, and remained far from the most effulgent horizon.}
\end{word}

\newpage

\begin{word}
\ayat{ای دوست}{\heading{O Friend!}{}} \\ \ayat{
در روضۀِ قلبْ

جز گُلِ عشق مکار
}{In the garden of thy heart plant naught but the rose of love,} \\ \ayat{
و از ذيلِ بلبلِ حبّ و شوق

دست مدار
}{and from the nightingale of affection and desire loosen not thy hold.} \\ \ayat{
مُصاحِبَّتِ اَبرار را

عنیمت دان
}{Treasure the companionship of the righteous} \\ \ayat{
و از مُرافِقَتِ اَشرار

دست و دل هر دو بردار
}{and eschew all fellowship with the ungodly.}
\end{word}

\newpage

\begin{word}
\ayat{ای پسرِ انصاف}{\heading{O Son of Justice!}{}} \\ \ayat{
کدام عاشق

که جز در وطنِ معشوق محلّ گيرد
}{Whither can a lover go but to the land of his beloved?} \\ \ayat{
و کدام طالب

که بی مطلوبْ راحت جوید
}{and what seeker findeth rest away from his heart's desire?} \\ \ayat{
عاشقِ صادق را

حيات در وصال است

و موت در فَراق
}{To the true lover reunion is life, and separation is death.} \\ \ayat{
صدرشان از صبرْ خالی

و قلوبشان از اصطبارْ مقدّس
}{His breast is void of patience and his heart hath no peace.} \\ \ayat{
از صد هزار جان درگُذَرَند

و بکویِ جانان شتابند
}{A myriad lives he would forsake to hasten to the abode of his beloved.}
\end{word}

\newpage

\begin{word}
\ayat{ای پسرِ خاک}{\heading{O Son of Dust!}{}} \\ \ayat{
براستی ميگویم
}{Verily I say unto thee:} \vspace{-1ex}\\ \ayat{
غافلترينِ عباد كسی است
}{Of all men the most negligent is he} \vspace{-1ex}\\ \ayat{
كه در قولْ مجادله نمايد

و بر برادرِ خود تَفَوُّق جويد
}{that disputeth idly and seeketh to advance himself over his brother.} \\ \ayat{
بگو ای برادران

باعمالْ خود را بيارائيد

نه باقوال
}{Say, O brethren! Let deeds, not words, be your adorning.}
\end{word}

\newpage

\begin{word}
\ayat{ای پسرانِ ارض}{\heading{O Son of Earth!}{}} \\ \ayat{
براستی بدانيد
}{Know, verily,} \vspace{-1ex}\\ \ayat{
قلبی كه در آن

شائبۀِ حَسَدْ باقی باشد
}{the heart wherein the least remnant of envy yet lingers,} \\ \ayat{
البتّه

بجبروتِ باقیِ من در نيايد
}{shall never attain My everlasting dominion,} \\ \ayat{
و از ملكوتِ تقديسِ من

رَوائِحِ قُدس نشنود
}{nor inhale the sweet savors of holiness breathing from My kingdom of
  sanctity.}
\end{word}

\newpage

\begin{word}
\ayat{ای پسرِ حبّ}{\heading{O Son of Love!}{}} \\ \ayat{
از تو تا رفرفِ اِمتِناعِ قرب

و سِدرِۀِ اِرتِفاعِ عِشق

قَدمی فاصله
}{Thou art but one step away from the glorious heights above and from the
  celestial tree of love.} \\ \ayat{
قَدمِ اوّلِ بردار

و قَدمِ ديگر بر عالمِ قِدم گُذار

و در سُرادقِ خُلد وارد شو
}{Take thou one pace and with the next advance into the immortal realm and
  enter the pavilion of eternity.} \vspace{-1ex}\\ \ayat{
پس بشنو آنچه

از قلمِ عِزّ نُزول يافت
}{Give ear then to that which hath been revealed by the pen of glory.}
\end{word}

\newpage

\begin{word}
\ayat{ای پسرِ عزّ}{\heading{O Son of Glory!}{}} \\ \ayat{
در سَبيلِ قُدس چالاک شو
}{Be swift in the path of holiness,} \vspace{-1ex}\\ \ayat{
و بر اَفلاکِ اُنس قَدَم گُذار
}{and enter the heaven of communion with Me.} \vspace{-1ex}\\ \ayat{
قلب را بصَيقَلِ روحْ پاک كن
}{Cleanse thy heart with the burnish of the spirit,} \vspace{-1ex}\\ \ayat{
و آهنگِ ساحتِ لولاک نما
}{and hasten to the court of the Most High.}
\end{word}

\newpage

\begin{word}
\ayat{ای سايۀِ نابود}{\heading{O Fleeting Shadow!}{}} \\ \ayat{
از مَدارِجِ ذُلِّ وَهْم بِگُذَر

و بِمَعارِجِ عزِّ يقين اَندَرآ
}{Pass beyond the baser stages of doubt and rise to the exalted heights of
  certainty.} \vspace{-1ex}\\ \ayat{
چشمِ حق بگشا
}{Open the eye of truth,} \vspace{-1ex}\\ \ayat{
تا جمالِ مبين بينی

و تَبَارَکَ ٱَللّٰهُ أَحْسَنُ ٱلخَالِقِينْ

گوئی
}{that thou mayest behold the veilless Beauty and exclaim: Hallowed be the
  Lord, the most excellent of all creators!}
\end{word}

\newpage

\begin{word}
\ayat{ای پسرِ هوی}{\heading{O Son of Desire!}{}} \\ \ayat{
براستی بشنو
}{Give ear unto this:} \vspace{-1ex}\\ \ayat{
چشمِ فانیْ

جمالِ باقی نَشِناسد
}{Never shall mortal eye recognize the everlasting Beauty,} \vspace{-1ex}\\ \ayat{
و دلِ مُرده

جز بِگُلِ پژمرده مشغول نشود
}{nor the lifeless heart delight in aught but in the withered bloom.} \\ \ayat{
زيرا كه هر قرينی

قرينِ خود را جويد

و بِجنِسِ خود اُنس گيرد
}{For like seeketh like, and taketh pleasure in the company of its kind.}
\end{word}

\newpage

\begin{word}
\ayat{ای پسرِ تُراب}{\heading{O Son of Dust!}{11}} \\ \ayat{
كور شو

تا جمالَم بينی
}{Blind thine eyes, that thou mayest behold My beauty;} \\ \ayat{
و كَر شو

تا لَحنْ و صَوْتِ مَليحَم را شنوی
}{stop thine ears, that thou mayest hearken unto the sweet melody of
  My voice;} \vspace{-1ex}\\ \ayat{
و جاهِل شو

تا از علْمَمْ نَصيب بری
}{empty thyself of all learning, that thou mayest partake of
  My knowledge;} \vspace{-1ex}\\ \ayat{
و فقير شو

تا از بحرِ غَنایِ لايَزالَم

قسمتِ بيزَوال برداری
}{and sanctify thyself from riches, that thou mayest obtain a lasting
  share from the ocean of My eternal wealth.}
\end{word}

\begin{word}
\ayat{
كور شو يعنی

از مشاهدۀ غَيرِ جمالِ من
}{Blind thine eyes, that is, to all save My beauty;} \vspace{-1ex}\\ \ayat{
و كر شو يعنی

از استماعِ كلامِ غَيرِ من
}{stop thine ears to all save My word;} \\ \ayat{
و جاهل شو يعنی

از سَوای عِلمِ من
}{empty thyself of all learning save the knowledge of Me;} \\ \ayat{
تا با چشمِ پاک

و دلِ طَيِّب

و گوشِ لطيف

بِساحَتِ قُدسَم درآئی
}{that with a clear vision, a pure heart and an attentive ear thou mayest
  enter the court of My holiness.}
\end{word}

\newpage

\begin{word}
\ayat{ای صاحبِ دو چشم}{\heading{O Man of Two Visions!}{}} \\ \ayat{
چشمی بَربَند

و چشمی بَرگُشا
}{Close one eye and open the other.} \\ \ayat{
بربند يعنی

از عالم و عالميان
}{Close one to the world and all that is therein,} \vspace{-1ex}\\ \ayat{
برگُشا يعنی

بِجمالِ قدسِ جانان
}{and open the other to the hallowed beauty of the Beloved.}
\end{word}

\newpage

\begin{word}
\ayat{ای پسرانِ من}{\heading{O My Children!}{}} \\ \ayat{
ترسم كه
}{I fear lest,} \vspace{-1ex}\\ \ayat{
از نغمۀِ ورقاء فَيض نبُرده
}{bereft of the melody of the dove of heaven,} \vspace{-1ex}\\ \ayat{
بِديارِ فنا راجِع شَويد
}{ye will sink back to the shades of utter loss,} \vspace{-1ex}\\ \ayat{
و جمالِ گُل نديده
}{and, never having gazed upon the beauty of the rose,} \vspace{-1ex}\\ \ayat{
بِآب و گِل باز گرديد
}{return to water and clay.}
\end{word}

\newpage

\begin{word}
\ayat{ای دوستان}{\heading{O Friends!}{}} \\ \ayat{
بِجمالِ فانی

از جمالِ باقی مَگُذَريد
}{Abandon not the everlasting beauty for a beauty that must die,} \\ \ayat{
و بِخاكدانِ تُرابی

دِل مبنديد
}{and set not your affections on this mortal world of dust.}
\end{word}

\pagebreak

\begin{word}
\ayat{ای پسر روح}{\heading{O Son of Spirit!}{}} \\ \ayat{
وقتی آيد
}{The time cometh,} \vspace{-1ex}\\ \ayat{
که بلبلِ قدسْ معنوی

از بيانِ اسرارْ معانی

ممنوع شود
}{when the nightingale of holiness will no longer unfold the inner
  mysteries} \\ \ayat{
و جميع از نغمهِ رَحمانی

و نِدایِ سبحانی

ممنوع گرديد
}{and ye will all be bereft of the celestial melody and of the voice from on
  high.}
\end{word}

\pagebreak

\begin{word}
\ayat{ای جوهر غفلت}{\heading{O Essence of Negligence!}{}} \\ \ayat{
دريغ که
}{} \vspace{-1ex}\\ \ayat{
صد هزار لسانِ معنوی

در لسانی ناطق
}{Myriads of mystic tongues find utterance in one speech,} \\ \ayat{
و صد هزار معانیِ غيبی

در لحنی ظاهر
}{and myriads of hidden mysteries are revealed in a single melody;} \\ \ayat{
و لکن گوشی نه تا بشنود
}{yet, alas, there is no ear to hear,} \vspace{-1ex}\\ \ayat{
و قلبی نه تا حرفی بيابد
}{nor heart to understand.}
\end{word}

\pagebreak

\begin{word}
\ayat{ای همگنان}{\heading{O Comrades!}{}} \\ \ayat{
ابوابِ لا مکان باز گشته
}{The gates that open on the Placeless stand wide} \vspace{-1ex}\\ \ayat{
و ديارِ جانان

از دَمِ عاشقان زينت يافته
}{and the habitation of the loved one is adorned with the lovers'
  blood,} \\ \ayat{
و جميع از اين شهرِ روحانی

محروم مانده اند

الّا قليلی
}{yet all but a few remain bereft of this celestial city,} \\ \ayat{
و از آن قليل
}{and even of these few,} \vspace{-1ex}\\ \ayat{
هم با قلبِ طاهر

و نفسِ مقدّس مشهود نگشت

الّا اَقَلِّ قليلی
}{none but the smallest handful hath been found with a pure heart and
  sanctified spirit.}
\end{word}

\pagebreak

\begin{word}
\ayat{ای اهلِ فردوس برين}{\heading{O ye Dwellers in the Highest Paradise!}{}} \\ \ayat{
اهلِ يقين را اخبار نمائيد

که در فضای قدس

قربِ رضوان

روضۀِ جديدی ظاهر گشته
}{Proclaim unto the children of assurance that within the realms of holiness, nigh unto the celestial paradise, a new
  garden hath appeared,} \\ \ayat{
و جميعِ اهلِ عالين

و هَياکِلِ خُلدِ برين

طائِفِ حَولِ آن گشته اند
}{round which circle the denizens of the realm on high and the immortal
  dwellers of the exalted paradise.} \\ \ayat{
پس جهدی نمائيد

تا بآن مقام درآئيد
}{Strive, then, that ye may attain that station,} \\ \ayat{
و حقائقِ اسرارِ عشق را

از شَقايِقَش جوئيد

و جميع حکمتهای بالِغِۀِ احديّه را

از اَثمارِ باقيه‌اش بيابيد
}{that ye may unravel the mysteries of love from its wind-flowers and learn
  the secret of divine and consummate wisdom from its eternal fruits.} \\
\ayat{
قُرَّت أَبْصَارُ الَّذِينَ

هُم دَخَلُوا فِيِه آمِنِِين
}{Solaced are the eyes of them that enter and abide therein!}
\end{word}

\pagebreak

\begin{word}
\ayat{ای دوستان من}{\heading{O My Friends!}{}} \\ \ayat{
آيا فراموش کرده ايد

آن صبح صادق رشنی را
}{Have ye forgotten that true and radiant morn,} \\ \ayat{
که در ظلّ شجره انيسا

که در فردوس اعظم غرس شده

جميع در آن فضای قدس مبارک

نزد من حاضر بوديد
}{when in those hallowed and blessed surroundings ye were all gathered in My
  presence beneath the shade of the tree of life, which is planted in the
  all-glorious paradise?} \\ \ayat{
و بسه کلمه طيّبه تکلّم فرمودم

و جميع آن کلماترا

شنيده و مدهوش گشتيد

و آن کلمات اين بود
}{Awe-struck ye listened as I gave utterance to these three most holy
  words:}
\end{word}

\pagebreak

\begin{word}
\ayat{ای دوستان}{O friends!} \\ \ayat{
رضای خود را بر رضای من

اختيار مکنيد
}{Prefer not your will to Mine,} \\ \ayat{
و آنچه برای شما نخواهم

هرگز مخواهيد
}{never desire that which I have not desired for you,} \\ \ayat{
و با دلهای مرده

که بآمال و آرزو آلوده شده

نزد من ميائيد
}{and approach Me not with lifeless hearts, defiled with worldly desires and
  cravings.} \\ \ayat{
اگر صدر را مقدّس کنيد
}{Would ye but sanctify your souls,} \vspace{-1ex}\\ \ayat{
حال آن صحرا

و آن فضا را

بنظر در آريد
}{ye would at this present hour recall that place and those
  surroundings,} \\ \ayat{
و بيان من بر همه شما

معلوم شود
}{and the truth of My utterance should be made evident unto all of you.}
\end{word}

\pagebreak

\begin{word}
\ayat{
در سَطرِ هشتم

از اَسطَرِ قدس

که در لوحِ پنجُم

از فردوس است ميفرمايد
}{In the eighth of the most holy lines, in the fifth Tablet of Paradise, He
  saith: }
\end{word}

\pagebreak

\begin{word}
\ayat{ای مردگان فِراشِ غفلت}{\heading{O ye that are Lying as Dead on the Couch
    of Heedlessness!}{}} \\ \ayat{
قرنها گُذشت

و عُمرِ گِرانمايه را

بِاِنتِها رسانده ايد
}{Ages have passed and your precious lives are well-nigh ended,} \\ \ayat{
و نَفَس پاکی از شما

بساحتِ قدسِ ما نيامد
}{yet not a single breath of purity hath reached Our court of holiness from
  you.} \vspace{-1ex}\\ \ayat{
در اَبحُرِ شِرک مُستَغَرَقِيد

و کَلِمِهِ توحيد بر زبان ميرانيد
}{Though immersed in the ocean of misbelief, yet with your lips ye profess the
  one true faith of God.} \vspace{-1ex}\\ \ayat{
مَبغُوضِ مرا

محبوبِ خود دانسته ايد

و دشمنِ مرا

دوستِ خود گرفته ايد
}{Him whom I abhor ye have loved, and of My foe ye have made a
  friend.}
\end{word}

\pagebreak

\begin{word}
\ayat{
و در ارضِ من

بکمال خُرَّمی و سرور

مَشْی مينمائيد
}{Notwithstanding, ye walk on My earth complacent and self-satisfied,} \\ \ayat{
و غافل از آنکه زمينِ من

از تو بيزار است
}{heedless that My earth is weary of you} \\ \ayat{
و اشيای ارض

از تو در گُريز
}{and everything within it shunneth you.} \\ \ayat{
اگر فی الجمله بَصَر بگشائی
}{Were ye but to open your eyes,} \vspace{-1ex}\\ \ayat{
صد هزار حُزن را

از اين سرور خوشتر دانی
}{ye would, in truth, prefer a myriad griefs unto this joy,} \\ \ayat{
و فنا را از اين حيات

نيکوتر شُمَری
}{and would count death itself better than this life.}
\end{word}

\pagebreak

\begin{word}
\ayat{ای خاکِ مُتَحَرِّک}{\heading{O Moving Form of Dust!}{}} \\ \ayat{
من بتو مأنوسم

و تو از من مأيوس
}{I desire communion with thee, but thou wouldst put no trust in Me.} \\ \ayat{
سيفِ عصيان

شجرهِ اميدِ تُرا بُريده
}{The sword of thy rebellion hath felled the tree of thy hope.} \\ \ayat{
و در جميعِ حال بتو نزديکم

و تو در جميعِ احوال از من دور
}{At all times I am near unto thee, but thou art ever far from Me.} \\ \ayat{
و من عزّتِ بيزوال

برای تو اختيار نِمودم

و تو ذلّتِ بی منتهیٰ

برای خود پسنديدی
}{Imperishable glory I have chosen for thee, yet boundless shame thou hast
  chosen for thyself.} \\ \\ \ayat{
آخر تا وقت باقی مانده

رُجوع کن

و فُرصت را مگذار
}{While there is yet time, return, and lose not thy chance.}
\end{word}

\pagebreak

\begin{word}
\ayat{ای پسرِ هویٰ}{\heading{O Son of Desire!}{}} \\ \ayat{
اهلِ دانش و بينش سالها کوشيدند

و بِوِصالِ ذَی الجلال فائز نگشتند
}{The learned and the wise have for long years striven and failed to attain
  the presence of the All-Glorious;} \vspace{-1ex}\\ \ayat{
و عُمرها دويدند

و بِلِقایِ ذَی الجمال نرسيدند
}{they have spent their lives in search of Him, yet did not behold the beauty
  of His countenance.} \vspace{-1ex}\\ \ayat{
و تو نادويده بمنزل رسيده

و ناطَلَبيده بِمَطلَب واصل شدی
}{Thou without the least effort didst attain thy goal, and without search hast
  obtained the object of thy quest.} \vspace{-1ex}\\ \ayat{
و بعد از جميعِ اين مقام و رتبه
}{} \vspace{-1ex}\\ \ayat{
بحجابِ نفسِ خود

چُنان مُحتَجِب ماندی
}{Yet, notwithstanding, thou didst remain so wrapt in the veil of
  self,} \\ \ayat{
که چشمت بجمالِ دوست نَيُفتاد

و دستت بدامنِ يار نرسيد
}{that thine eyes beheld not the beauty of the Beloved, nor did thy hand touch
  the hem of His robe.} \vspace{-1ex}\\ \ayat{
فَتَعَجَّبُوا مِن ذلکَ يا اُولِی الأَبصَار
}{Ye that have eyes, behold and wonder.}
\end{word}

\pagebreak

\begin{word}
\ayat{ای اهل ديار عشق}{\heading{O Dwellers in the City of Love!}{}} \vspace{-1ex}\\ \ayat{
شمع باقی را اَرياحِ فانی

اِحاطه نموده
}{Mortal blasts have beset the everlasting candle,} \vspace{-1ex}\\ \ayat{
و جمالِ غُلامِ روحانی

در غُبارِ تيرهِ ظلمانی مستور مانده
}{and the beauty of the celestial Youth is veiled in the darkness of
  dust.} \vspace{-1ex}\\ \ayat{
سلطانِ سَلاطينِ عشق

در دستِ رَعايایِ ظُلم مظلوم
}{The chief of the monarchs of love is wronged by the people of
  tyranny} \vspace{-1ex}\\ \ayat{
و حَمامِهِ قدسی

در دستِ جُغدان گرفتار
}{and the dove of holiness lies prisoned in the talons of owls.} \vspace{-1ex}\\ \ayat{
جميعِ اهلِ سُرادقِ ابهیٰ

و ملأ اَعلیٰ

نُوحِه و نُدبِه مينمايند
}{The dwellers in the pavilion of glory and the celestial concourse bewail and
  lament,} \\ \ayat{
و شما در کمالِ راحت

در ارضِ غفلت اقامت نموده ايد
}{while ye repose in the realm of negligence,} \vspace{-1ex}\\ \ayat{
و خود را هم از دوستانِ خالص

مَحسوب داشته ايد
}{and esteem yourselves as of the true friends.} \vspace{-1ex}\\ \ayat{
فَباطِل ما أنتُم تَظَنُّون
}{How vain are your imaginings!}
\end{word}

\pagebreak

\begin{word}
\ayat{ای جُهَلایِ مَعروف بِعِلم}{\heading{O ye that are Foolish, yet have a Name to be Wise!}{}} \\ \ayat{
چرا در ظاهر دَعوِیِ شَبانی کنيد
}{Wherefore do ye wear the guise of shepherds,} \vspace{-1ex}\\ \ayat{
و در باطِن ذِئِبِ اَغنامِ من شده ايد
}{when inwardly ye have become wolves, intent upon My flock?} \vspace{-1ex}\\ \ayat{
مَثَلِ شما

مِثلِ ستارهِ قبل از صُبح است
}{Ye are even as the star, which riseth ere the dawn,} \\ \ayat{
که در ظاهر دُرّيّ و روشن است
}{and which, though it seem radiant and luminous,} \vspace{-1ex}\\ \ayat{
و در باطن سببِ اِضلال و هَلاکَت

کارِوانهایِ مدينه و ديارِ من است
}{leadeth the wayfarers of My city astray into the paths of perdition.}
\end{word}

%todo%
% \pagebreak

% \begin{word}
% \ayat{ای بظاهر آراسته و بباطن کاسته}{\heading{O ye Seeming Fair yet Inwardly Foul!}{}} \\ \ayat{
% مَثَلِ شما مثل آب تلخ صافی است که کمال لطافت و صفا از آن در ظاهر مشهود شود چون بدست صرّاف ذائقه احديّه افتد قطره ای از آن را قبول نفرمايد
% بلی تجلّی آفتاب در تراب و مرآت هر دو موجود ولکن از فَرْقَدان تا ارض فرق دان بلکه فرق بی منتهی در ميان
% }{
%   Ye are like clear but bitter water, which to outward seeming is crystal pure
%   but of which, when tested by the divine Assayer, not a drop is accepted.
%   Yea, the sun beam falls alike upon the dust and the mirror, yet differ they
%   in reflection even as doth the star from the earth: nay, immeasurable is the
%   difference!
% }
% \end{word}

% \pagebreak

% \begin{word}
% \ayat{ای دوست لسانی من}{\heading{O My Friend in Word!}{}} \\ \ayat{
% قدری تأمّل اختيار کن هرگز شنيده ای که يار و اغيار در قلبی بگنجد ؟ پس اغيار را بران تا جانان بمنزل خود در آيد
% }{
%   Ponder awhile. Hast thou ever heard that friend and foe should abide in one
%   heart? Cast out then the stranger, that the Friend may enter His home.
% }
% \end{word}

% \pagebreak

% \begin{word}
% \ayat{ای پسر خاک}{\heading{O Son of Dust!}{}} \\ \ayat{
% جميع آنچه در آسمانها و زمين است برای تو مقرّر داشتم مگر قلوب را که محلّ نزول تجلّی جمال و اجلال خود معيّن فرمودم
% و تو منزل و محلّ مرا بغير من گذاشتی چنانچه در هر زمان که ظهور قدس من آهنگ مکان خود نمود غير خود را يافت اغيار ديد و لا مکان بحرم جانان شتافت
% و مع ذلک ستر نمودم و سرّ نگشودم و خجلت ترا نپسنديدم
% }{
%   All that is in heaven and earth I have ordained for thee, except the human
%   heart, which I have made the habitation of My beauty and glory; yet thou
%   didst give My home and dwelling to another than Me; and whenever the
%   manifestation of My holiness sought His own abode, a stranger found He
%   there, and, homeless, hastened unto the sanctuary of the Beloved.
%   Notwithstanding I have concealed thy secret and desired not thy shame.
% }
% \end{word}

% \pagebreak

% \begin{word}
% \ayat{ای جوهر هوی}{\heading{O Essence of Desire!}{}} \\ \ayat{
% بسا سحرگاهان که از مشرق لا مکان بمکان تو آمدم و ترا در بستر راحت بغير خود مشغول يافتم و چون برق روحانی بغمام عزّ سلطانی رجوع نمودم و در مکامن قرب خود نزد جنود قدس اظهار نداشتم
% }{
%   At many a dawn have I turned from the realms of the Placeless unto thine
%   abode, and found thee on the bed of ease busied with others than Myself.
%   Thereupon, even as the flash of the spirit, I returned to the realms of
%   celestial glory and breathed it not in My retreats above unto the hosts of
%   holiness.
% }
% \end{word}

% \pagebreak

% \begin{word}
% \ayat{ای پسر جود}{\heading{O Son of Bounty!}{}} \\ \ayat{
% در باديه های عدم بودی و ترا بمدد تراب امر در عالم ملک ظاهر نمودم و جميع ذرّات ممکنات و حقائق کائنات را بر تربيت تو گماشتم چنانچه قبل از خروج از بطن امّ دو چشمه شير منير برای تو مقرّر داشتم و چشمها برای حفظ تو گماشتم و حبّ ترا در قلوب القا نمودم و بصرف جود ترا در ظلّ رحمتم پروردم و از جوهر فضل و رحمت ترا حفظ فرمودم
% و مقصود از جميع اين مراتب آن بود که بجبروت باقی ما درآئی و قابل بخششهای غيبی ما شوی و تو غافل چون بثمر آمدی از تمامی نعيمم غفلت نمودی و بگمان باطل خود پرداختی بقسمی که بالمرّه فراموش نمودی و از باب دوست بايوان دشمن مقرّ يافتی و مسکن نمودی
% }{
%   Out of the wastes of nothingness, with the clay of My command I made thee to
%   appear, and have ordained for thy training every atom in existence and the
%   essence of all created things. Thus, ere thou didst issue from thy mother's
%   womb, I destined for thee two founts of gleaming milk, eyes to watch over
%   thee, and hearts to love thee. Out of My loving-kindness, 'neath the shade
%   of My mercy I nurtured thee, and guarded thee by the essence of My grace and
%   favor. And My purpose in all this was that thou mightest attain My
%   everlasting dominion and become worthy of My invisible bestowals. And yet
%   heedless thou didst remain, and when fully grown, thou didst neglect all My
%   bounties and occupied thyself with thine idle imaginings, in such wise that
%   thou didst become wholly forgetful, and, turning away from the portals of
%   the Friend didst abide within the courts of My enemy.
% }
% \end{word}

% \pagebreak

% \begin{word}
% \ayat{ای بنده دنيا}{\heading{O Bond Slave of the World!}{}} \\ \ayat{
% در سحرگاهان نسيم عنايت من بر تو مرور نمود و ترا در فراش غفلت خفته يافت و بر حال تو گريست و باز گشت
% }{
%   Many a dawn hath the breeze of My loving-kindness wafted over thee and found
%   thee upon the bed of heedlessness fast asleep. Bewailing then thy plight it
%   returned whence it came.
% }
% \end{word}

% \pagebreak

% \begin{word}
% \ayat{ای پسر ارض}{\heading{O Son of Earth!}{}} \\ \ayat{
% اگر مرا خواهی جز مرا مخواه و اگر اراده جمالم داری چشم از عالميان بردار زيرا که اراده من و غير من چون آب و آتش در يک دل و قلب نگنجد
% }{
%   Wouldst thou have Me, seek none other than Me; and wouldst thou gaze upon My
%   beauty, close thine eyes to the world and all that is therein; for My will
%   and the will of another than Me, even as fire and water, cannot dwell
%   together in one heart.
% }
% \end{word}

% \pagebreak

% \begin{word}
% \ayat{ای برادر من}{\heading{O Befriended Stranger!}{}} \\ \ayat{
% از لسان شکرينم کلمات نازنينم شنو و از لب نمکينم سلسبيل قدس معنوی بياشام يعنی تخمهای حکمت لدنّيم را در ارض طاهر قلب بيفشان و بآب يقين آبش ده تا سنبلات علم و حکمت من سر سبز از بلده طيّبه انبات نمايد
% }{
%   The candle of thine heart is lighted by the hand of My power, quench it not
%   with the contrary winds of self and passion. The healer of all thine ills is
%   remembrance of Me, forget it not. Make My love thy treasure and cherish it
%   even as thy very sight and life.
% }
% \end{word}

% \pagebreak

% \begin{word}
% \ayat{ای بيگانهء با يگانه}{\heading{O My Brother!}{}} \\ \ayat{
% شمع دلت برافروخته دست قدرت منست آن را ببادهای مخالف نفس و هوی خاموش مکن و طبيب جميع علّتهای تو ذکر منست فراموشش منما
% حبّ مرا سرمايه خود کن و چون بصر و جان عزيزش دار
% }{
%   Hearken to the delightsome words of My honeyed tongue, and quaff the stream
%   of mystic holiness from My sugar-shedding lips. Sow the seeds of My divine
%   wisdom in the pure soil of thy heart, and water them with the water of
%   certitude, that the hyacinths of My knowledge and wisdom may spring up fresh
%   and green in the sacred city of thy heart.
% }
% \end{word}

% \pagebreak

% \begin{word}
% \ayat{ای اهل رضوان من}{\heading{O Dwellers of My Paradise!}{}} \\ \ayat{
% نهال محبّت و دوستی شما را در روضه قدس رضوان بيد ملاطفت غرس نمودم و بنيسان مرحمت آبش دادم حال نزديک بثمر رسيده جهدی نمائيد تا محفوظ ماند و بنار امل و شهوت نسوزد
% }{
%   With the hands of loving-kindness I have planted in the holy garden of
%   paradise the young tree of your love and friendship, and have watered it
%   with the goodly showers of My tender grace; now that the hour of its
%   fruiting is come, strive that it may be protected, and be not consumed with
%   the flame of desire and passion.
% }
% \end{word}

% \pagebreak

% \begin{word}
% \ayat{ای دوستان من}{\heading{O My Friends!}{}} \\ \ayat{
% سراج ضلالت را خاموش کنيد و مشاعل باقيهء هدايت در قلب و دل برافروزيد که عنقريب صرّافان وجود در پيشگاه حضور معبود جز تقوای خالص نپذيرند و غير عمل پاک قبول ننمايند
% }{
%   Quench ye the lamp of error, and kindle within your hearts the everlasting
%   torch of divine guidance. For ere long the assayers of mankind shall, in the
%   holy presence of the Adored, accept naught but purest virtue and deeds of
%   stainless holiness.
% }
% \end{word}

% \pagebreak

% \begin{word}
% \ayat{ای پسر تراب}{\heading{O Son of Dust!}{}} \\ \ayat{
% حکمای عباد آنانند که تا سمع نيابند لب نگشايند چنانچه ساقی تا طلب نبيند ساغر نبخشد و عاشق تا بجمال معشوق فائز نشود از جان نخروشد
% پس بايد حبّه های حکمت و علم را در ارض طيّبه قلب مبذول داريد و مستور نمائيد تا سنبلات حکمت الهی از دِل برآيد نه از گِل
% }{
%   The wise are they that speak not unless they obtain a hearing, even as the
%   cup-bearer, who proffereth not his cup till he findeth a seeker, and the
%   lover who crieth not out from the depths of his heart until he gazeth upon
%   the beauty of his beloved. Wherefore sow the seeds of wisdom and knowledge
%   in the pure soil of the heart, and keep them hidden, till the hyacinths of
%   divine wisdom spring from the heart and not from mire and clay.
% }
% \end{word}

% \pagebreak

% \begin{word}
% \ayat{
% در سطر اوّل لوح مذکور و مسطورست و در سرادق حفظ اللّه مستور
% }{
%   In the first line of the Tablet it is recorded and written, and within the
%   sanctuary of the tabernacle of God is hidden:
% }
% \end{word}

% \pagebreak

% \begin{word}
% \ayat{ای بنده من}{\heading{O My Servant!}{}} \\ \ayat{
% ملک بی زوال را بانزالی از دست منه و شاهنشهی فردوس را بشهوتی از دست مده اينست کوثر حيوان که از معين قلم رحمن ساری گشته طوبی للشّاربين
% }{
%   Abandon not for that which perisheth an everlasting dominion, and cast not
%   away celestial sovereignty for a worldly desire. This is the river of
%   everlasting life that hath flowed from the well-spring of the pen of the
%   merciful; well is it with them that drink!
% }
% \end{word}

% \pagebreak

% \begin{word}
% \ayat{ای پسر روح}{\heading{O Son of Spirit!}{}} \\ \ayat{
% قفس بشکن و چون همای عشق بهوای قدس پرواز کن و از نَفْس بگذر و با نَفَس رحمانی در فضای قدس ربّانی بيارام
% }{
%   Burst thy cage asunder, and even as the phoenix of love soar into the
%   firmament of holiness. Renounce thyself and, filled with the spirit of
%   mercy, abide in the realm of celestial sanctity.
% }
% \end{word}

% \pagebreak

% \begin{word}
% \ayat{ای پسر رماد}{\heading{O Offspring of Dust!}{}} \\ \ayat{
% براحت يومی قانع مشو و از راحت بيزوال باقيه مگذر و گلشن باقی عيش جاودان را بگلخن فانی ترابی تبديل منما
% از زندان بصحراهای خوش جان عروج کن و از قفس امکان برضوان دلکش لا مکان بخرام
% }{
%   Be not content with the ease of a passing day, and deprive not thyself of
%   everlasting rest. Barter not the garden of eternal delight for the dust-heap
%   of a mortal world. Up from thy prison ascend unto the glorious meads above,
%   and from thy mortal cage wing thy flight unto the paradise of the Placeless.
% }
% \end{word}

% \pagebreak

% \begin{word}
% \ayat{ای بنده من}{\heading{O My Servant!}{}} \\ \ayat{
% از بند ملک خود را رهائی بخش و از حبس نفس خود را آزاد کن وقت را غنيمت شمر زيرا که اين وقت را ديگر نبينی و اين زمان را هرگز نيابی
% }{
%   Free thyself from the fetters of this world, and loose thy soul from the
%   prison of self. Seize thy chance, for it will come to thee no more.
% }
% \end{word}

% \pagebreak

% \begin{word}
% \ayat{ای فرزند کنيز من}{\heading{O Son of My Handmaid!}{}} \\ \ayat{
% اگر سلطنت باقی بينی البتّه بکمال جدّ از ملک فانی درگذری و لکن ستر آنرا حکمتهاست و جلوه اين را رمزها جز افئده پاک ادراک ننمايد
% }{
%   Didst thou behold immortal sovereignty, thou wouldst strive to pass from
%   this fleeting world. But to conceal the one from thee and to reveal the
%   other is a mystery which none but the pure in heart can comprehend.
% }
% \end{word}

% \pagebreak

% \begin{word}
% \ayat{ای بنده من}{\heading{O My Servant!}{}} \\ \ayat{
% دل را از غلّ پاک کن و بی حسد ببساط قدس احد بخرام
% }{
%   Purge thy heart from malice and, innocent of envy, enter the divine court of
%   holiness.
% }
% \end{word}

% \pagebreak

% \begin{word}
% \ayat{ای دوستان من}{\heading{O My Friends!}{}} \\ \ayat{
% در سبيل رضای دوست مشی نمائيد و رضای او در خلق او بوده و خواهد بود يعنی دوست بی رضای دوست خود در بيت او وارد نشود و در اموال او تصرّف ننمايد و رضای خود را بر رضای او ترجيح ندهد و خود را در هيچ امری مقدّم نشمارد
% فتفکّروا فی ذلک يا اولی الافکار
% }{
%   Walk ye in the ways of the good pleasure of the Friend, and know that His
%   pleasure is in the pleasure of His creatures. That is: no man should enter
%   the house of his friend save at his friend's pleasure, nor lay hands upon
%   his treasures nor prefer his own will to his friend's, and in no wise seek
%   an advantage over him. Ponder this, ye that have insight!
% }
% \end{word}

% \pagebreak

% \begin{word}
% \ayat{ای رفيق عرشی}{\heading{O Companion of My Throne!}{}} \\ \ayat{
% بد مشنو و بد مبين و خود را ذليل مکن و عويل برميار
% يعنی بد مگو تا نشنوی و عيب مردم را بزرگ مدان تا عيب تو بزرگ ننمايد و ذلّت نفسی مپسند تا ذلّت تو چهره نگشايد
% پس با دل پاک و قلب طاهر و صدر مقدّس و خاطر منزّه در ايّام عمر خود که اقلّ از آنی محسوبست فارغ باش تا بفراغت از اين جسد فانی بفردوس معانی راجع شوی و در ملکوت باقی مقرّ يابی
% }{
%   Hear no evil, and see no evil, abase not thyself, neither sigh and weep.
%   Speak no evil, that thou mayest not hear it spoken unto thee, and magnify
%   not the faults of others that thine own faults may not appear great; and
%   wish not the abasement of anyone, that thine own abasement be not exposed.
%   Live then the days of thy life, that are less than a fleeting moment, with
%   thy mind stainless, thy heart unsullied, thy thoughts pure, and thy nature
%   sanctified, so that, free and content, thou mayest put away this mortal
%   frame, and repair unto the mystic paradise and abide in the eternal kingdom
%   for evermore.
% }
% \end{word}

% \pagebreak

% \begin{word}
% \ayat{وای وای ای عاشقان هوای نفسانی}{\heading{Alas!  Alas!  O Lovers of Worldly Desire!}{}} \\ \ayat{
% از معشوق روحانی چون برق گذشته ايد و بخيال شيطانی دل محکم بسته ايد
% ساجد خياليد و اسم آن را حقّ گذاشته ايد و ناظر خاريد و نام آن را گل گذارده ايد نه نَفَس فارغی از شما برآمد و نه نسيم انقطاعی از رياض قلوبتان وزيد
% نصايح مشفقه محبوبرا بباد داده ايد و از صفحه دل محو نموده ايد و چون بهآئم در سبزه زار شهوت و امل تعيّش مينمائيد
% }{
%   Even as the swiftness of lightning ye have passed by the Beloved One, and
%   have set your hearts on satanic fancies. Ye bow the knee before your vain
%   imagining, and call it truth. Ye turn your eyes towards the thorn, and name
%   it a flower. Not a pure breath have ye breathed, nor hath the breeze of
%   detachment been wafted from the meadows of your hearts. Ye have cast to the
%   winds the loving counsels of the Beloved and have effaced them utterly from
%   the tablet of your hearts, and even as the beasts of the field, ye move and
%   have your being within the pastures of desire and passion.
% }
% \end{word}

% \pagebreak

% \begin{word}
% \ayat{ای برادران طريق}{\heading{O Brethren in the Path!}{}} \\ \ayat{
% چرا از ذکر نگار غافل گشته ايد و از قرب حضرت يار دور مانده ايد ؟ و شما بهوای خود بجدال مشغول گشته ايد
% روايح قدس ميوزد و نسائم جود در هبوب و کلّ بزکام مبتلی شده ايد و از جميع محروم مانده ايد
% زهی حسرت بر شما و علی الّذين هم يمشون علی أعقابکم و علی أثر أقدامکم هم يمرّون
% }{
%   Wherefore have ye neglected the mention of the Loved One, and kept remote
%   from His holy presence? The essence of beauty is within the peerless
%   pavilion, set upon the throne of glory, whilst ye busy yourselves with idle
%   contentions. The sweet savors of holiness are breathing and the breath of
%   bounty is wafted, yet ye are all sorely afflicted and deprived thereof. Alas
%   for you and for them that walk in your ways and follow in your footsteps!
% }
% \end{word}

% \pagebreak

% \begin{word}
% \ayat{ای پسران آمال}{\heading{O Children of Desire!}{}} \\ \ayat{
% جامه غرور را از تن بر آريد و ثوب تکبّر از بدن بيندازيد در سطر سيّم از اسطر قدس که در لوح ياقوتی از قلم خفی ثبت شده اين است
% }{
%   Put away the garment of vainglory, and divest yourselves of the attire of
%   haughtiness. In the third of the most holy lines writ and recorded in the
%   Ruby Tablet by the pen of the unseen this is revealed:
% }
% \end{word}

% \pagebreak

% \begin{word}
% \ayat{ای برادران}{\heading{O Brethren!}{}} \\ \ayat{
% با يکديگر مدارا نمائيد و از دنيا دل برداريد بعزّت افتخار منمائيد و از ذلّت ننگ مداريد قسم بجمالم که کلّ را از تراب خلق نمودم و البتّه بخاک راجع فرمايم
% }{
%   Be forbearing one with another and set not your affections on things below.
%   Pride not yourselves in your glory, and be not ashamed of abasement. By My
%   beauty! I have created all things from dust, and to dust will I return them
%   again.
% }
% \end{word}

% \pagebreak

% \begin{word}
% \ayat{ای پسران تراب}{\heading{O Children of Dust!}{}} \\ \ayat{
% اغنيا را از ناله سحرگاهی فقرا اخبار کنيد که مبادا از غفلت بهلاکت افتند و از سدره دولت بی نصيب مانند
% الکرم و الجود من خصالی فهنيئاً لمن تزيّن بخصالی
% }{
%   Tell the rich of the midnight sighing of the poor, lest heedlessness lead
%   them into the path of destruction, and deprive them of the Tree of Wealth.
%   To give and to be generous are attributes of Mine; well is it with him that
%   adorneth himself with My virtues.
% }
% \end{word}

% \pagebreak

% \begin{word}
% \ayat{ای ساذج هوی}{\heading{O Quintessence of Passion!}{}} \\ \ayat{
% حرص را بايد گذاشت و بقناعت قانع شد
% زيرا که لازال حريص محروم بوده و قانع محبوب و مقبول
% }{
%   Put away all covetousness and seek contentment; for the covetous hath ever
%   been deprived, and the contented hath ever been loved and praised.
% }
% \end{word}

% \pagebreak

% \begin{word}
% \ayat{ای پسر کنيز من}{\heading{O Son of My Handmaid!}{}} \\ \ayat{
% در فقر اضطراب نشايد و در غنا اطمينان نبايد
% هر فقری را غنا در پی و هر غنا را فنا از عقب و لکن فقر از ماسوی اللّه نعمتی است بزرگ حقير مشماريد زيرا که در غايت آن غنای باللّه رخ بگشايد
% و در اين مقام أنتم الفقرآء مستور و کلمه مبارکه و اللّه هو الغنيّ چون صبح صادق از افق قلب عاشق ظاهر و باهر و هويدا و آشکار شود و بر عرش غنا متمکّن گردد و مقرّ يابد
% }{
%   Be not troubled in poverty nor confident in riches, for poverty is followed
%   by riches, and riches are followed by poverty. Yet to be poor in all save
%   God is a wondrous gift, belittle not the value thereof, for in the end it
%   will make thee rich in God, and thus thou shalt know the meaning of the
%   utterance, ``In truth ye are the poor,'' and the holy words, ``God is the
%   all-possessing,'' shall even as the true morn break forth gloriously
%   resplendent upon the horizon of the lover's heart, and abide secure on the
%   throne of wealth.
% }
% \end{word}

% \pagebreak

% \begin{word}
% \ayat{ای پسران غفلت و هوی}{\heading{O Children of Negligence and Passion!}{}} \\ \ayat{
% دشمن مرا در خانه من راه داده ايد و دوست مرا از خود رانده ايد چنانچه حبّ غير مرا در دل منزل داده ايد
% بشنويد بيان دوست را و برضوانش اقبال نمائيد
% دوستان ظاهر نظر بمصلحت خود يکديگر را دوست داشته و دارند و لکن دوست معنوی شما را لاجل شما دوست داشته و دارد بلکه مخصوص هدايت شما بلايای لا تحصی قبول فرموده
% بچنين دوست جفا مکنيد و بکويش بشتابيد
% اينست شمس کلمه صدق و وفا که از افق اصبع مالک اسماء اشراق فرموده
% افتحوا آذانکم لاصغاء کلمة اللّه المهيمن القيّوم
% }{
%   Ye have suffered My enemy to enter My house and have cast out My friend, for
%   ye have enshrined the love of another than Me in your hearts. Give ear to
%   the sayings of the Friend and turn towards His paradise. Worldly friends,
%   seeking their own good, appear to love one the other, whereas the true
%   Friend hath loved and doth love you for your own sakes; indeed He hath
%   suffered for your guidance countless afflictions. Be not disloyal to such a
%   Friend, nay rather hasten unto Him. Such is the daystar of the word of truth
%   and faithfulness, that hath dawned above the horizon of the pen of the Lord
%   of all names. Open your ears that ye may hearken unto the word of God, the
%   Help in peril, the Self-existent.
% }
% \end{word}

% \pagebreak

% \begin{word}
% \ayat{ای مغروران باموال فانيه}{\heading{O ye that Pride Yourselves on Mortal Riches!}{}} \\ \ayat{
% بدانيد که غنا سدّيست محکم ميان طالب و مطلوب و عاشق و معشوق هرگز غنی بر مقرّ قرب وارد نشود و بمدينه رضا و تسليم در نيايد مگر قليلی
% پس نيکوست حال آن غنی که غنا از ملکوت جاودانی منعش ننمايد و از دولت ابدی محرومش نگرداند
% قسم باسم اعظم که نور آن غنی اهل آسمان را روشنی بخشد چنانچه شمس اهل زمين را
% }{
%   Know ye in truth that wealth is a mighty barrier between the seeker and his
%   desire, the lover and his beloved. The rich, but for a few, shall in no wise
%   attain the court of His presence nor enter the city of content and
%   resignation. Well is it then with him, who, being rich, is not hindered by
%   his riches from the eternal kingdom, nor deprived by them of imperishable
%   dominion. By the Most Great Name! The splendor of such a wealthy man shall
%   illuminate the dwellers of heaven even as the sun enlightens the people of
%   the earth!
% }
% \end{word}

% \pagebreak

% \begin{word}
% \ayat{ای اغنيای ارض}{\heading{O ye Rich Ones on Earth!}{}} \\ \ayat{
% فقرا امانت منند در ميان شما
% پس امانت مرا درست حفظ نمائيد و براحت نفس خود تمام نپردازيد
% }{
%   The poor in your midst are My trust; guard ye My trust, and be not intent
%   only on your own ease.
% }
% \end{word}

% \pagebreak

% \begin{word}
% \ayat{ای فرزند هوی}{\heading{O Son of Passion!}{}} \\ \ayat{
% از آلايش غنا پاک شو و با کمال آسايش در افلاک فقر قدم گذار تا خمر بقا از عين فنا بياشامی
% }{
%   Cleanse thyself from the defilement of riches and in perfect peace advance
%   into the realm of poverty; that from the well-spring of detachment thou
%   mayest quaff the wine of immortal life.
% }
% \end{word}

% \pagebreak

% \begin{word}
% \ayat{ای پسر من}{\heading{O My Son!}{}} \\ \ayat{
% صحبت اشرار غم بيفزايد و مصاحبت ابرار زنگ دل بزدايد
% من أراد ان يأنس مع اللّه فليأنس مع احبّائه و من أراد ان يسمع کلام اللّه فليسمع کلمات اصفيائه
% }{
%   The company of the ungodly increaseth sorrow, whilst fellowship with the
%   righteous cleanseth the rust from off the heart. He that seeketh to commune
%   with God, let him betake himself to the companionship of His loved ones; and
%   he that desireth to hearken unto the word of God, let him give ear to the
%   words of His chosen ones.
% }
% \end{word}

% \pagebreak

% \begin{word}
% \ayat{زينهار ای پسر خاک}{\heading{O Son of Dust!}{}} \\ \ayat{
% با اشرار الفت مگير و مؤانست مجو که مجالست اشرار نور جان را بنار حسبان تبديل نمايد
% }{
%   Beware! Walk not with the ungodly and seek not fellowship with him, for such
%   companionship turneth the radiance of the heart into infernal fire.

% }
% \end{word}

% \pagebreak

% \begin{word}
% \ayat{ای پسر کنيز من}{\heading{O Son of My Handmaid!}{}} \\ \ayat{
% اگر فيض روح القدس طلبی با احرار مصاحب شو
% زيرا که ابرار جام باقی از کف ساقی خلد نوشيده اند و قلب مردگان را چون
% }{
%   Wouldst thou seek the grace of the Holy Spirit, enter into fellowship with
%   the righteous, for he hath drunk the cup of eternal life at the hands of the
%   immortal Cup-bearer and even as the true morn doth quicken and illumine the
%   hearts of the dead.
% }
% \end{word}

% \pagebreak

% \begin{word}
% \ayat{ای غافلان}{\heading{O Heedless Ones!}{}} \\ \ayat{
% گمان مبريد که اسرار قلوب مستور است بلکه بيقين بدانيد که بخطّ جلی مسطور گشته و در پيشگاه حضور مشهود
% }{
%   Think not the secrets of hearts are hidden, nay, know ye of a certainty that
%   in clear characters they are engraved and are openly manifest in the holy
%   Presence.
% }
% \end{word}

% \pagebreak

% \begin{word}
% \ayat{ای دوستان}{\heading{O Friends!}{}} \\ \ayat{
% براستی ميگويم که جميع آنچه در قلوب مستور نموده ايد نزد ما چون روز واضح و ظاهر و هويداست و لکن ستر آنرا سبب جود و فضل ماست نه استحقاق شما
% }{
%   Verily I say, whatsoever ye have concealed within your hearts is to Us open
%   and manifest as the day; but that it is hidden is of Our grace and favor,
%   and not of your deserving.
% }
% \end{word}

% \pagebreak

% \begin{word}
% \ayat{ای پسر انسان}{\heading{O Son of Man!}{}} \\ \ayat{
% شبنمی از ژرف دريای رحمت خود بر عالميان مبذول داشتم و احدی را مقبل نيافتم زيرا که کلّ از خمر باقی لطيف توحيد بماء کثيف نبيد اقبال نموده اند و از کأس جمال باقی بجام فانی قانع شده اند
% فبئس ما هُم به يقنعون
% }{
%   A dewdrop out of the fathomless ocean of My mercy I have shed upon the
%   peoples of the world, yet found none turn thereunto, inasmuch as every one
%   hath turned away from the celestial wine of unity unto the foul dregs of
%   impurity, and, content with mortal cup, hath put away the chalice of
%   immortal beauty. Vile is that wherewith he is contented.
% }
% \end{word}

% \pagebreak

% \begin{word}
% \ayat{ای پسر خاک}{\heading{O Son of Dust!}{}} \\ \ayat{
% از خمر بی مثال محبوب لا يزال چشم مپوش و بخمر کدره فانيه چشم مگشا از دست ساقی احديّه کأوس باقيه برگير تا همه هوش شوی و از سروش غيب معنوی شنوی بگو ای پست فطرتان از شراب باقی قدسم چرا بآب فانی رجوع نموديد
% }{
%   Turn not away thine eyes from the matchless wine of the immortal Beloved,
%   and open them not to foul and mortal dregs. Take from the hands of the
%   divine Cup-bearer the chalice of immortal life, that all wisdom may be
%   thine, and that thou mayest hearken unto the mystic voice calling from the
%   realm of the invisible. Cry aloud, ye that are of low aim! Wherefore have ye
%   turned away from My holy and immortal wine unto evanescent water?
% }
% \end{word}

% \pagebreak

% \begin{word}
% \ayat{بگو ای اهل ارض}{\heading{O ye Peoples of the World!}{}} \\ \ayat{
% براستی بدانيد که بلای ناگهانی شما را در پی است و عقاب عظيمی از عقب
% گمان مبريد که آنچه را مرتکب شديد از نظر محو شده
% قسم بجمالم که در الواح زبرجدی از قلم جلی جميع اعمال شما ثبت گشته
% }{
%   Know, verily, that an unforeseen calamity followeth you, and grievous
%   retribution awaiteth you. Think not that which ye have committed hath been
%   effaced in My sight. By My beauty! All your doings hath My pen graven with
%   open characters upon tablets of chrysolite.
% }
% \end{word}

% \pagebreak

% \begin{word}
% \ayat{ای ظالمان ارض}{\heading{O Oppressors on Earth!}{}} \\ \ayat{
% از ظلم دست خود را کوتاه نمائيد که قسم ياد نموده‌ام از ظلم احدی نگذرم و اين عهدی است که در لوح محفوظ محتوم داشتم و بخاتم عزّ مختوم
% }{
%   Withdraw your hands from tyranny, for I have pledged Myself not to forgive
%   any man's injustice. This is My covenant which I have irrevocably decreed in
%   the preserved tablet and sealed with My seal.
% }
% \end{word}

% \pagebreak

% \begin{word}
% \ayat{ای عاصيان}{\heading{O Rebellious Ones!}{}} \\ \ayat{
% بردباری من شما را جری نمود و صبر من شما را بغفلت آورد که در سبيلهای مهلک خطرناک بر مراکب نار نفس بی باک ميرانيد گويا مرا غافل شمرده ايد و يا بی خبر انگاشته ايد
% }{
%   My forbearance hath emboldened you and My long-suffering hath made you
%   negligent, in such wise that ye have spurred on the fiery charger of passion
%   into perilous ways that lead unto destruction. Have ye thought Me heedless
%   or that I was unaware?
% }
% \end{word}

% \pagebreak

% \begin{word}
% \ayat{ای مهاجران}{\heading{O Emigrants!}{}} \\ \ayat{
% لسان مخصوص ذکر منست بغيبت ميالائيد و اگر نفس ناری غلبه نمايد بذکر عيوب خود مشغول شويد نه بغيبت خلق من زيرا که هر کدام از شما بنفس خود اَبصَر و اعرفيد از نفوس عباد من
% }{
%   The tongue I have designed for the mention of Me, defile it not with
%   detraction. If the fire of self overcome you, remember your own faults and
%   not the faults of My creatures, inasmuch as every one of you knoweth his own
%   self better than he knoweth others.
% }
% \end{word}

% \pagebreak

% \begin{word}
% \ayat{ای پسران وهم}{\heading{O Children of Fancy!}{}} \\ \ayat{
% بدانيد چون صبح نورانی از افق قدس صمدانی بردمد البتّه اسرار و اعمال شيطانی که در ليل ظلمانی معمول شده ظاهر شود و بر عالميان هويدا گردد
% }{
%   Know, verily, that while the radiant dawn breaketh above the horizon of
%   eternal holiness, the satanic secrets and deeds done in the gloom of night
%   shall be laid bare and manifest before the peoples of the world.
% }
% \end{word}

% \pagebreak

% \begin{word}
% \ayat{ای گياه خاک}{\heading{O Weed that Springeth out of Dust!}{}} \\ \ayat{
% چگونه است که با دست آلوده بشکر مباشرت جامه خود ننمائی و با دل آلوده بکثافت شهوت و هوی معاشرتم را جوئی و بممالک قدسم راه خواهی ؟ هيهات هيهات عمّا أنتم تريدون
% }{
%   Wherefore have not these soiled hands of thine touched first thine own
%   garment, and why with thine heart defiled with desire and passion dost thou
%   seek to commune with Me and to enter My sacred realm? Far, far are ye from
%   that which ye desire.
% }
% \end{word}

% \pagebreak

% \begin{word}
% \ayat{ای پسران آدم}{\heading{O Children of Adam!}{}} \\ \ayat{
% کلمه طيّبه و اعمال طاهره مقدّسه بسمآء عزّ احديّه صعود نمايد
% جهد کنيد تا اعمال از غبار ريا و کدورت نفس و هوی پاک شود و بساحت عزّ قبول درآيد
% چه که عنقريب صرّافان وجود در پيشگاه حضور معبود جز تقوای خالص نپذيرند و غير عمل پاک قبول ننمايند
% اينست آفتاب حکمت و معانی که از افق فم مشيّت ربّانی اشراق فرمود طوبی للمقبلين
% }{
%   Holy words and pure and goodly deeds ascend unto the heaven of celestial
%   glory. Strive that your deeds may be cleansed from the dust of self and
%   hypocrisy and find favor at the court of glory; for ere long the assayers of
%   mankind shall, in the holy presence of the Adored One, accept naught but
%   absolute virtue and deeds of stainless purity. This is the daystar of wisdom
%   and of divine mystery that hath shone above the horizon of the divine will.
%   Blessed are they that turn thereunto.
% }
% \end{word}

% \pagebreak

% \begin{word}
% \ayat{ای پسر عيش}{\heading{O Son of Worldliness!}{}} \\ \ayat{
% خوش ساحتی است ساحت هستی اگر اندر آئی و نيکو بساطی است بساط باقی اگر از ملک فانی برتر خرامی و مليح است نشاط مستی اگر ساغر معانی از يد غلام الهی بياشامی اگر باين مراتب فائز شوی از نيستی و فنا و محنت و خطا فارغ گردی
% }{
%   Pleasant is the realm of being, wert thou to attain thereto; glorious is the
%   domain of eternity, shouldst thou pass beyond the world of mortality; sweet
%   is the holy ecstasy if thou drinkest of the mystic chalice from the hands of
%   the celestial Youth. Shouldst thou attain this station, thou wouldst be
%   freed from destruction and death, from toil and sin.
% }
% \end{word}

% \pagebreak

% \begin{word}
% \ayat{ای دوستان من}{\heading{O My Friends!}{}} \\ \ayat{
% ياد آوريد آن عهدی را که در جبل فاران که در بقعه مبارکه زمان واقع شده با من نموده ايد و ملأ اعلی و اصحاب مدين بقا را بر آن عهد گواه گرفتم و حال احديرا بر آن عهد قائم نمی بينم البتّه غرور و نافرمانی آن را از قلوب محو نموده بقسميکه اثری از آن باقی نمانده و من دانسته صبر نمودم و اظهار نداشتم
% }{
%   Call ye to mind that covenant ye have entered into with Me upon Mount Paran,
%   situate within the hallowed precincts of Zaman. I have taken to witness the
%   concourse on high and the dwellers in the city of eternity, yet now none do
%   I find faithful unto the covenant. Of a certainty pride and rebellion have
%   effaced it from the hearts, in such wise that no trace thereof remaineth.
%   Yet knowing this, I waited and disclosed it not.
% }
% \end{word}

% \pagebreak

% \begin{word}
% \ayat{ای بنده من}{\heading{O My Servant!}{}} \\ \ayat{
% مثل تو مثل سيف پر جوهری است که در غلاف تيره پنهان باشد و باين سبب قدر آن بر جوهريان مستور ماند
% پس از غلاف نفس و هوی بيرون آی تا جوهر تو بر عالميان هويدا و روشن آيد
% }{
%   Thou art even as a finely tempered sword concealed in the darkness of its
%   sheath and its value hidden from the artificer's knowledge. Wherefore come
%   forth from the sheath of self and desire that thy worth may be made
%   resplendent and manifest unto all the world.
% }
% \end{word}

% \pagebreak

% \begin{word}
% \ayat{ای دوست من}{\heading{O My Friend!}{}} \\ \ayat{
% تو شمس سمآء قدس منی خود را بکسوف دنيا ميالای
% حجاب غفلت را خرق کن تا بی پرده و حجاب از خلف سحاب بدر آئی و جميع موجودات را بخلعت هستی بيارائی
% }{
%   Thou art the daystar of the heavens of My holiness, let not the defilement
%   of the world eclipse thy splendor. Rend asunder the veil of heedlessness,
%   that from behind the clouds thou mayest emerge resplendent and array all
%   things with the apparel of life.
% }
% \end{word}

% \pagebreak

% \begin{word}
% \ayat{ای ابنآء غرور}{\heading{O Children of Vainglory!}{}} \\ \ayat{
% بسلطنت فانيه ايّامی از جبروت باقی من گذشته و خود را باسباب زرد و سرخ می آرائيد و بدين سبب افتخار مينمائيد
% قسم بجمالم که جميع را در خيمه يکرنگ تراب درآورم و همه اين رنگهای مختلفه را از ميان بردارم مگر کسانيکه برنگ من درآيند و آن تقديس از همهء رنگها است
% }{
%   For a fleeting sovereignty ye have abandoned My imperishable dominion, and
%   have adorned yourselves with the gay livery of the world and made of it your
%   boast. By My beauty! All will I gather beneath the one-colored covering of
%   the dust and efface all these diverse colors save them that choose My own,
%   and that is purging from every color.
% }
% \end{word}

% \pagebreak

% \begin{word}
% \ayat{ای ابنآء غفلت}{\heading{O Children of Negligence!}{}} \\ \ayat{
% بپادشاهی فانی دل مبنديد و مسرور مشويد
% مثل شما مثل طير غافلی است که بر شاخه باغی در کمال اطمينان بسرايد و بغتةً صيّاد اجل او را بخاک اندازد ديگر از نغمه و هيکل و رنگ او اثری باقی نماند
% پس پند گيريد ای بندگان هوی
% }{
%   Set not your affections on mortal sovereignty and rejoice not therein. Ye
%   are even as the unwary bird that with full confidence warbleth upon the
%   bough; till of a sudden the fowler Death throws it upon the dust, and the
%   melody, the form and the color are gone, leaving not a trace. Wherefore take
%   heed, O bondslaves of desire!
% }
% \end{word}

% \pagebreak

% \begin{word}
% \ayat{ای فرزند کنيز من}{\heading{O Son of My Handmaid!}{}} \\ \ayat{
% لازال هدايت باقوال بوده و اين زمان بافعال گشته
% يعنی بايد جميع افعال قدسی از هيکل انسانی ظاهر شود چه که در اقوال کلّ شريکند و لکن افعال پاک و مقدّس مخصوص دوستان ماست
% پس بجان سعی نمائيد تا بافعال از جميع ناس ممتاز شويد
% کذلک نصحناکم فی لوح قدس منير
% }{
%   Guidance hath ever been given by words, and now it is given by deeds. Every
%   one must show forth deeds that are pure and holy, for words are the property
%   of all alike, whereas such deeds as these belong only to Our loved ones.
%   Strive then with heart and soul to distinguish yourselves by your deeds. In
%   this wise We counsel you in this holy and resplendent tablet.
% }
% \end{word}

% \pagebreak

% \begin{word}
% \ayat{ای پسر انصاف}{\heading{O Son of Justice!}{}} \\ \ayat{
% در ليل جمال هيکل بقا از عقبه زمرّدی وفا بسدره منتهی رجوع نمود

% و گريست گريستنی که جميع ملأ عالين و کروبين از ناله او گريستند

% و بعد از سبب نوحه و ندبه استفسار شد مذکور داشت که حسب الأمر در عقبه وفا منتظر ماندم و رائحه وفا از اهل ارض نيافتم و بعد آهنگ رجوع نمودم ملحوظ افتاد که حمامات قدسی چند در دست کلاب ارض مبتلی شده اند
% در اين وقت حوريّه الهی از قصر روحانی بی ستر و حجاب دويد و سؤال از اسامی ايشان نمود و جميع مذکور شد الّا اسمی از اسمآء
% و چون اصرار رفت حرف اوّل اسم از لسان جاری شد اهل غرفات از مکامن عزّ خود بيرون دويدند و چون بحرف دوم رسيد جميع بر تراب ريختند در آن وقت ندا از مکمن قرب رسيد زياده بر اين جايز نه انّا کنّا شهدآء علی ما فعلوا و حينئذٍ کانوا يفعلون
% }{
%   In the night-season the beauty of the immortal Being hath repaired from the
%   emerald height of fidelity unto the Sadratu'l-Muntahá, and wept with such a
%   weeping that the concourse on high and the dwellers of the realms above
%   wailed at His lamenting. Whereupon there was asked, Why the wailing and
%   weeping? He made reply: As bidden I waited expectant upon the hill of
%   faithfulness, yet inhaled not from them that dwell on earth the fragrance of
%   fidelity. Then summoned to return I beheld, and lo! certain doves of
%   holiness were sore tried within the claws of the dogs of earth. Thereupon
%   the Maid of heaven hastened forth unveiled and resplendent from Her mystic
%   mansion, and asked of their names, and all were told but one. And when
%   urged, the first letter thereof was uttered, whereupon the dwellers of the
%   celestial chambers rushed forth out of their habitation of glory. And whilst
%   the second letter was pronounced they fell down, one and all, upon the dust.
%   At that moment a voice was heard from the inmost shrine: ``Thus far and no
%   farther.'' Verily We bear witness to that which they have done and now are
%   doing.
% }
% \end{word}

% \pagebreak

% \begin{word}
% \ayat{ای فرزند کنيز من}{\heading{O Son of My Handmaid!}{}} \\ \ayat{
% از لسان رحمن سلسبيل معانی بنوش و از مشرق بيان سبحان اشراق انوار شمس تبيان من غير ستر و کتمان مشاهده نما
% تخمهای حکمت لدنّيم را در ارض طاهر قلب بيفشان و بآب يقين آبش ده تا سنبلات علم و حکمت من سر سبز از بلده طيّبه انبات نمايد
% }{
%   Quaff from the tongue of the merciful the stream of divine mystery, and
%   behold from the dayspring of divine utterance the unveiled splendor of the
%   daystar of wisdom. Sow the seeds of My divine wisdom in the pure soil of the
%   heart, and water them with the waters of certitude, that the hyacinths of
%   knowledge and wisdom may spring up fresh and green from the holy city of the
%   heart.
% }
% \end{word}

% \pagebreak

% \begin{word}
% \ayat{ای پسر هوی}{\heading{O Son of Desire!}{}} \\ \ayat{
% تا کی در هوای نفسانی طيران نمائی ؟ پر عنايت فرمودم تا در هوای قدس معانی پرواز کنی نه در فضای وهم شيطانی
% شانه مرحمت فرمودم تا گيسوی مشکينم شانه نمائی نه گلويم بخراشی
% }{
%   How long wilt thou soar in the realms of desire? Wings have I bestowed upon
%   thee, that thou mayest fly to the realms of mystic holiness and not the
%   regions of satanic fancy. The comb, too, have I given thee that thou mayest
%   dress My raven locks, and not lacerate My throat.
% }
% \end{word}

% \pagebreak

% \begin{word}
% \ayat{ای بندگان من}{\heading{O My Servants!}{}} \\ \ayat{
% شما اشجار رضوان منيد بايد باثمار بديعه منيعه ظاهر شويد تا خود و ديگران از شما منتفع شوند
% لذا بر کلّ لازم که بصنايع و اکتساب مشغول گردند
% اينست اسباب غنا يا اولی الألباب و انّ الأمور معلّقة باسبابها و فضل اللّه يغنيکم بها
% و اشجار بی ثمار لايق نار بوده و خواهد بود
% }{
%   Ye are the trees of My garden; ye must give forth goodly and wondrous
%   fruits, that ye yourselves and others may profit therefrom. Thus it is
%   incumbent on every one to engage in crafts and professions, for therein lies
%   the secret of wealth, O men of understanding! For results depend upon means,
%   and the grace of God shall be all-sufficient unto you. Trees that yield no
%   fruit have been and will ever be for the fire.
% }
% \end{word}

% \pagebreak

% \begin{word}
% \ayat{ای بنده من}{\heading{O My Servant!}{}} \\ \ayat{
% پست‌ترين ناس نفوسی هستند که بی ثمر در ارض ظاهرند و فی الحقيقه از اموات محسوبند بلکه اموات از آن نفوس معطّله مهمله ارجح عند اللّه مذکور
% }{
%   The basest of men are they that yield no fruit on earth. Such men are verily
%   counted as among the dead, nay better are the dead in the sight of God than
%   those idle and worthless souls.
% }
% \end{word}

% \pagebreak

% \begin{word}
% \ayat{ای بنده من}{\heading{O My Servant!}{}} \\ \ayat{
% بهترين ناس آنانند که باقتراف تحصيل کنند و صرف خود و ذوی القربی نمايند حبّاً للّه ربّ العالمين
% }{
%   The best of men are they that earn a livelihood by their calling and spend
%   upon themselves and upon their kindred for the love of God, the Lord of all
%   worlds.
% }
% \end{word}

% \pagebreak

% \begin{word}
% \ayat{
% عروسِ معانیِ بديعه
% }{The mystic and wondrous Bride,} \\ \ayat{
% که ورای پرده‌های بيان

% مستور و پنهان بود
% }{hidden ere this beneath the veiling of utterance,} \\ \ayat{
% بعنايت الهی و الطاف ربّانی

% چون شعاع منير جمال دوست

% ظاهر و هويدا شد
% }{hath now, by the grace of God and His divine favor, been made manifest
%   even as the resplendent light shed by the beauty of the Beloved.} \\ \ayat{
% شهادت ميدهم ای دوستان
% }{I bear witness, O friends!} \\ \ayat{
% که نعمت تمام

% و حجّت کامل

% و برهان ظاهر

% و دليل ثابت آمد
% }{that the favor is complete, the argument fulfilled, the proof manifest
%   and the evidence established.} \\ \ayat{
% ديگر تا همّت شما

% از مراتب انقطاع

% چه ظاهر نمايد
% }{Let it now be seen what your endeavors in the path of detachment will
%   reveal.} \\ \ayat{
% کذلک تمّت النّعمة عليکم

% و علی من فی السّموات و الأرضين
% }{In this wise hath the divine favor been fully vouchsafed unto you and
%   unto them that are in heaven and on earth.} \\ \ayat{
% و الحمد للّه ربّ العالمين
% }{All praise to God, the Lord of all Worlds.}
% \end{word}

\end{document}
