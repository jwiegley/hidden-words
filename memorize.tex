% -*- bidi-paragraph-direction: left-to-right -*-

\documentclass[11pt]{article}

\usepackage[margin=1in,top=1.25in,bottom=1.25in]{geometry}
\usepackage{soul}
\usepackage{tabularx}
\usepackage{fontspec}
\usepackage{xunicode}
\usepackage{setspace}
\usepackage{arabxetex}

\setuldepth{sh}

\setmainfont[Ligatures=TeX]{GaramondPremrPro}[
  Path           = /Users/johnw/Library/Fonts/ ,
  Extension      = .otf ,
  BoldFont       = *-Smbd ,
  ItalicFont     = *-It ,
  BoldItalicFont = *-SmbdIt
]

\newfontfamily\arabicfont{Scheherazade}[
  Path        = /Users/johnw/Library/Fonts/ ,
  Extension   = .ttf ,
  UprightFont = *RegOT ,
  Script      = Arabic
]

\newfontfamily\headwordfont[Ligatures=TeX]{Georgia}[
  Path           = /Users/johnw/Library/Fonts/ ,
  Extension      = .ttf ,
  UprightFont    = * ,
  BoldFont       = * Bold ,
  ItalicFont     = * Italic ,
  BoldItalicFont = * Bold Italic ,
  Script         = Arabic
]

\newenvironment{orig}
  {\begin{farsi}[voc]}
  {\end{farsi}}

\newenvironment{trans}
  {\Large\begin{spacing}{1.2}\raggedright}
  {\end{spacing}}

\newenvironment{word}
  {\begin{tabular}[t]{p{2.75in}@{\hspace{3em}}p{2.75in}}}
  {\end{tabular}}

\newcommand{\ayat}[2]{\begin{orig}#1\end{orig} & \begin{trans}#2\end{trans}}
\newcommand{\heading}[2]{\textsc{\textbf{#1}} % \ (\##2)
}
\newcommand{\define}[3]{\textfarsi[voc]{\Huge
    \textbf{#1}}\hspace{3mm}{\headwordfont \large
    \textit{#2}}\hspace{3mm}{\Large #3} \\[3ex]}
\newcommand{\fulldefine}[3]{\textfarsi[voc]{\Huge
    \textbf{#1}}\hspace{3mm}{\headwordfont \large
    \textit{#2}}\vspace{-1ex}\begin{quote}\Large #3\end{quote}\vspace{1ex}}

\title{
\Huge
\vspace*{2in}
Selections from \\
the Sacred Writings \\
\vspace{.25in}
\fontsize{48}{36}
\begin{arab}
\end{arab}
\vspace{1in}}
\author{\LARGE Bahá’u’lláh and `Abdu'l-Bahá}
\date{}

\begin{document}

\maketitle
\thispagestyle{empty}

\newpage

\fontsize{24}{32}

\begin{word}
\ayat{
اِلهَا پَروَردِگارا مَحبوبا مَقصُودا
}{O God, O God, my Beloved, the Goal of my Desire!} \\ \ayat{
به تو آمده ام و از تو می طلبم

آنچه را كه سببِ بخششِ تو است
}{I stand before Thee and beseech Thee by reason of Thy forgiveness,} \\ \ayat{
توئی بحرِ جود و مالكِ وجود
}{O Thou Who art the Ocean of bounty and the King of existence,} \\ \ayat{
لازال

لحاظتِ علّتِ ظهورِ بخشش و عطا
}{O Thou Who hast caused both forgiveness and tenderness to appear.} \\ \ayat{
عبادِ خود را محروم منما
}{Deny not Thy servants} \\ \ayat{
و از بِساطِ قُدس و قُرب منع مفرما
}{and withhold them not from Thy holiness and nearness.} \\ \ayat{
توئی بخشنده و مهربان
}{Thou art the Forgiving and the Kind.} \\ \ayat{
لا اِلهَ اِلّا اَنتَ العَزيزُ المَنّان
}{No God is there but Thee, the Almighty, the Most Bountiful.}
\end{word}

\newpage

\begin{word}
\ayat{
ای احبّای حقّ
}{O ye the beloved of the one true God!} \vspace{-1ex}\\ \ayat{
از مَفازَۀِ ضَیِّقِۀِ نفس و هویٰ
}{Pass beyond the narrow retreats of your evil and corrupt desires,} \vspace{-1ex}\\ \ayat{
به فَضاهایِ مُقَدَّسِۀِ اَحَدِیِّه بِشتابید
}{and advance into the vast immensity of the realm of God,} \vspace{-1ex}\\ \ayat{
و در حَدِیقِۀِ تَقدیس و تَنزیه

مَأویٰ گیرید
}{and abide ye in the meads of sanctity and of detachment,} \\ \ayat{
تا از نَفَحاتِ اَعمالِیِّه کُلّ بَرِیِّه

بشاطیِ عِزِّ اَحَدِیِّه تَوَجُّه نمایند
}{that the fragrance of your deeds may lead the whole of mankind to the ocean
  of God’s unfading glory.} \vspace{-1ex}\\ \ayat{
اَبَداً در اُمورِ دُنیا و مَا یَتَعَلَّقُ بها

و رُؤسایِ ظاهِرۀِ آن تَکَلُّم جایز نه
}{Forbear ye from concerning yourselves with the affairs of this world and all
  that pertaineth unto it, or from meddling with the activities of those who
  are its outward leaders.}
\end{word}

% \newpage

% \begin{minipage}[t]{0.48\textwidth}
% \define{صاحبان}{sá\d{h}ibán}{owners}
% \fulldefine{هدهد}{hudhud}{The hoopoe bird, often a guide in Sufi literature.}
% \end{minipage}

\newpage

\begin{word}
\ayat{
اللّه ابهی
}{} \vspace{-1ex}\\ \ayat{
ای متوجّه اِلی اللّه
}{O thou who art turning thy face towards God!} \vspace{-1ex}\\ \ayat{
چشم از جميعِ ماسِوی بَر بَند
}{Close thine eyes to all things else,} \vspace{-1ex}\\ \ayat{
و به مَلَکوتِ ابهی بَر گشا
}{and open them to the realm of the All-Glorious.} \vspace{-1ex}\\ \ayat{
آنچه خواهی از او خواه
}{Ask whatsoever thou wishest of Him alone;} \vspace{-1ex}\\ \ayat{
و آنچه طلبی از او طلب
}{seek whatsoever thou seekest from Him alone.} \vspace{-1ex}\\ \ayat{
به نظری

صد هزار حاجاتت روا نمايد
}{With a look He granteth a hundred thousand hopes,} \vspace{-1ex}\\ \ayat{
و به التفاتی

صد هزار درد بی درمان دوا کند
}{with a glance He healeth a hundred thousand incurable ills,} \vspace{-1ex}\\ \ayat{
و به انعطافی

زخم‌ها را مَرهم نهد
}{with a nod He layeth balm on every wound,} \vspace{-1ex}\\ \ayat{
و به نگاهی

دل‌ها را از قيدِ غم برهاند
}{with a glimpse He freeth the hearts from the shackles of grief.}
\end{word}

\newpage

\begin{word}
\ayat{
آنچه کند او کند
}{He doeth as He doeth,} \vspace{-1ex}\\ \ayat{
ما چه توانيم کرد
}{and what recourse have we?} \vspace{-1ex}\\ \ayat{
يَفعل ما يَشاء
}{He carrieth out His Will,} \vspace{-1ex}\\ \ayat{
و يَحکُمُ ما يُريد است
}{He ordaineth what He pleaseth.} \vspace{-1ex}\\ \ayat{
پس سَرِ تسليم نـِه
}{Then better for thee to bow down thy head in submission,} \vspace{-1ex}\\ \ayat{
و توکّل بر رَبِّ رحيم بـِه
}{and put thy trust in the All-Merciful Lord.} \vspace{-1ex}\\ \ayat{
والبهاء عليک

ع ع
}{}
\end{word}

\newpage

\begin{word}
\ayat{
حکایت آورده‌اند
}{The story is told} \vspace{-1ex}\\ \ayat{
که عارف الهی با عالم نحوی

همراه شدند و همراز گشتند
}{of a mystic knower who went on a journey with a learned grammarian for a
  companion.} \vspace{-1ex}\\ \ayat{
تا رسیدند بشاطی بحر العظمة
}{They came to the shore of the Sea of Grandeur.} \vspace{-1ex}\\ \ayat{
عارف بی ‌تأمّل توسّل فرموده

بر آب راند
}{The knower, putting his trust in God, straightway flung himself into the
  waves,} \vspace{-1ex}\\ \ayat{
و عالم نحو

چون نقشِ بر آب محو گشته

مبهوت ماند
}{but the grammarian stood bewildered and lost in thoughts that were as words
  traced upon the water.} \\ \ayat{
بانگ زد عارف

که چون عنان پیچیدی
}{The mystic called out to him, “Why dost thou not follow?”}
\end{word}

\newpage

\begin{word}
\ayat{
گفت
}{The grammarian answered,} \vspace{-1ex}\\ \ayat{
ای برادر چه کنم
}{“O brother, what can I do?} \vspace{-1ex}\\ \ayat{
چون پای رفتنم نیست

سَرْ نَهادن اولیٰ بود
}{As I dare not advance, I must needs go back again.”} \\ \ayat{
گفت
}{Then the mystic cried,} \vspace{-1ex}\\ \ayat{
آنچه از سیبویه و قولویه

اَخْذ نموده‌ئی

و یا از مطالب ابن حاجب

و ابن مالک

حَمْل فرموده ئی

بریز و از آب بگذر
}{“Cast aside what thou hast learned from Síbavayh and Qawlavayh, from
  Ibn-i-Hájib and Ibn-i-Málik, and cross the water!”\vspace{15ex}} \\ \ayat{
محو میباید نه نحو این را بدان
}{With renunciation, not with grammar’s rules, one must be armed:} \vspace{-1ex}\\ \ayat{
گر تو محوی بی خطر در آب ران
}{Be nothing, then, and cross this sea unharmed.}
\end{word}

\newpage

\begin{word}
\ayat{
هو الله

ای طالب ملکوت

به الطاف حضرت پروردگار

اميدوار باش

و از مَصائِبِ شَديدۀِ اين جهان

نااميد مگرد

الحمدلله

خدای مهربان داری

که طَبيبِ هر بيمار است

و غَمخوارِ هر مُبتَلا

پَناهِ يَتيمان است

و مُعين بيکَسان

و بينهايت مِهرِبان
}{}
\end{word}

\newpage

\begin{word}
\ayat{
اگر بدانی که قلب عبد البهاء

چه قَدر مهربان است

البتّه از شِدَّتِ فَرَح و سُرور

پَرواز نمائی

و فَريادِ واطوبىٰ

به اُوجِ آسمان رِسانی


والبهاو عَلَیکْ

عبد البهاء عبّاس
}{}
\end{word}

\newpage

% }{} \\ \ayat{
\begin{word}
\ayat{
زبانِ خِرَد میگوید
}{The Tongue of Wisdom proclaimeth:} \vspace{-1ex}\\ \ayat{
هر که دارایِ من نباشد

دارایِ هیچ نه
}{He that hath Me not is bereft of all things.} \vspace{-1ex}\\ \ayat{
از هر چه هست بگذرید

و مرا بیابید
}{Turn ye away from all that is on earth and seek none else but Me.} \\ \ayat{
منم آفتابِ بینش

و دریایِ دانش
}{I am the Sun of Wisdom and the Ocean of Knowledge.} \vspace{-1ex}\\ \ayat{
پژمردگان را تازه نمایم

و مردگان را زنده کنم
}{I cheer the faint and revive the dead.} \\ \ayat{
منم آن روشنائی

که راهِ دیده بنمایم
}{I am the guiding Light that illumineth the way.} \vspace{-1ex}\\ \ayat{
و منم شاهبازِ دستِ بینیاز
}{I am the royal Falcon on the arm of the Almighty.} \vspace{-1ex}\\ \ayat{
پرِ بستگان را بگشایم

و پرواز بیاموزم
}{I unfold the drooping wings of every broken bird and start it on its
  flight.}
\end{word}

\newpage

% }{} \\ \ayat{
\begin{word}
\ayat{
و بقدم یقین
}{Thus with steadfast steps} \vspace{-1ex}\\ \ayat{
در صراط حقّ الیقین

قدم گذاریم
}{we may tread the Path of certitude,} \\ \ayat{
که لعلّ نسیم رضا
}{that perchance the breeze that bloweth
from the meads of the good-pleasure of God} \vspace{-1ex}\\ \ayat{
از ریاض قبول الهی بوزد
}{may waft upon us the sweet savours
    of divine acceptance,} \vspace{-1ex}\\ \ayat{
و این فانیان را
}{and cause us, vanishing mortals that we are,} \vspace{-1ex}\\ \ayat{
بملکوت جاودانی رساند
}{to attain unto the Kingdom of everlasting glory.}
\end{word}

\newpage

\begin{word}
\ayat{
هو الله

رشحِ عما از جذبۀ ما میریزد
سرّ وفا از نغمۀ ما میریزد

از باد صَبا مُشگ خطا گشته پدید
وین نفخۀ خوش از جَعدِۀ ما میریزد

شمسِ طَراز از طَلعَت حَقّ کرده طُلوع
سِرّ حقیقت بین کَز وجهۀ ما میریزد

بحرِ صَفا از موجِ لِقا کرده خُروش
وین طُرفهْ عطا از جذبۀ ما میریزد

گنجینۀ حُبّ در سینۀ فا گشته نَهان
زین گَنجِ مُحَبَّت دُرِّ وفا ميريزد

بهجتِ مُل از نَظرِۀ گُل شد ظاهر
این رَمزِ ملیح از رَنِّۀ را میریزد

نُقرۀ ناقوری جذبۀ لاهوتی
این هر دو بیک نَفخِه از جَوِّ سَما میریزد

دَورِ اَنَا هُو از چهرۀ ما کرده بُروز
کَورِ هُوَ هُو از نَفخِۀ ما میریزد

کوثرِ حقّ از کاسۀ دل گشته هُویدا
وین ساغرِ شَهْد از لَعلِ بها میریزد

یومِ خدا از جِلوِۀ ربّ شد کامل
این نَغزِ حَدیث از غَنۀ طا میریزد

طَفحِ بهائی بین رشح عمائی بین
کِاین جُمله زِ یک نغمه از لَحنِ خدا میریزد

ماهیِ سرمد بین طَلعِ مُنَزَّه بین
صَدرِ مُمَرَّد بین کز عَرشِ علا میریزد

نَخلِۀ طوبیٰ بین رَنِّۀ ورقا بین
غَنِّۀ ابهی بین کَز لَمْعِ صَفا میریزد

آهنگِ عراقی بین دَفِّ حجازی بین
کَفِّ الهیٰ بین کز جذبه لا میریزد

طَلعَةِ لاهوتی بین حوریِ هاهوتی بین
جِلوِۀ ناسوتی بین کز سِرِّ عما میریزد

وَجۀ باقی بین چِهرۀ ساقی بین
رَقِّ زُجاجی بین کز کوبۀ ما میریزد

آتَشِ موسیٰ بین بَیضِۀ بَیضا بین
سینۀ سینا بین کز کفِّ سَنا میریزد

نالِۀ مَستان بین حالتِ بُستان بین
جذبۀ هَستان بین کز صحنِ لِقا میریزد

غُنچۀ هائی بین طُردۀ بائی بین
رَنِّۀ نائی بین کز کِلکِ بها میریزد

طَفحِ طَهور است این رَشح طَهور است این
غَن طُیور است این کز عِین فَنا میریزد
}{}
\end{word}

\end{document}
